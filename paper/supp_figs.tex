\documentclass{article}

\usepackage[top=3cm, bottom=3cm, left=3cm,right=3cm]{geometry}
\usepackage[colorinlistoftodos]{todonotes}
\usepackage{graphicx}
\usepackage{amssymb}
\usepackage{amsmath}
\usepackage{bbm}
\usepackage{todonotes}
\usepackage{pdflscape}
\usepackage{caption}
\usepackage{subcaption}
\usepackage[T1]{fontenc}
\usepackage[utf8]{inputenc}
\usepackage{authblk}
\usepackage{pdfpages}
\usepackage{setspace} 
\usepackage{booktabs}
\usepackage{longtable}
\usepackage{float}
\usepackage{tikz}
\usepackage[colorlinks=true,citecolor=blue, linkcolor=blue]{hyperref}
\usepackage{multirow}
\setlength{\tabcolsep}{5pt}
%%\setlength{\parindent}{0pt}
\usepackage[parfill]{parskip}
\renewcommand{\arraystretch}{1.5}

\renewcommand\Affilfont{\itshape\footnotesize}
\def\ci{\perp\!\!\!\perp}

\renewcommand\Affilfont{\itshape\footnotesize}
\linespread{1.5}

% \usepackage{lineno}
% \linenumbers

% Nature Bibliography style
\usepackage[backend=biber,style=nature]{biblatex}
\addbibresource{library.bib} 

%%%%%%%%%%%%%
%%% Title %%%
%%%%%%%%%%%%%
\title{Supplementary Material: District-level male medical and traditional circumcision coverage and unmet need in sub-Saharan Africa}

\author{}
\date{}

%%%%%%%%%%%%%%%%%%%%%%%%%%%%%%%%%%%%%%%%%%%%%%%%%%%%%%%%%%%%%%%%%%
%%%%%%%%%%%%%%%%%%%%%%%%%%%%%%%%%%%%%%%%%%%%%%%%%%%%%%%%%%%%%%%%%%
%%%%%%%%%%%%%%%%%%%%%%%%%%%%%%%%%%%%%%%%%%%%%%%%%%%%%%%%%%%%%%%%%%

\begin{document}

%%%%%%%%%%%%%%%%%%%%%%%%%%%%%%%%%%%%%%%%%%%%%%%%%%%%%%%%%%%%%%%%%%
%%%%%%%%%%%%%%%%%%%%%%%%%%%%%%%%%%%%%%%%%%%%%%%%%%%%%%%%%%%%%%%%%%
%%%%%%%%%%%%%%%%%%%%%%%%%%%%%%%%%%%%%%%%%%%%%%%%%%%%%%%%%%%%%%%%%%

\maketitle

\vspace{-1cm}

Patrick O'Toole\textsuperscript{1},
Matthew L. Thomas\textsuperscript{1,2},
Oliver Stevens\textsuperscript{1},
Kevin Lam\textsuperscript{1,3},
Katharine Kripke\textsuperscript{4},
Rachel Esra\textsuperscript{1},
Ian Wanyeki\textsuperscript{5},
Lycias Zembe\textsuperscript{5},
Jeffrey W. Eaton\textsuperscript{1,6} \\
\smallskip
  
\textbf{1} MRC Centre for Global Infectious Disease Analysis, School of Public Health, Imperial Colleg  e London, London, United Kingdom\\
\textbf{2} Department of Earth and Environmental Sciences, University of Manchester, Manchester, United Kingdom\\
\textbf{3} Department of Statistics, University of British Columbia, Vancouver, Canada\\
\textbf{4} Avenir Health, Washington, District of Columbia, United States of America\\
\textbf{5} Joint United Nations Programme on HIV/AIDS (UNAIDS), Geneva, Switzerland\\
\textbf{6} Center for Communicable Disease Dynamics, Department of Epidemiology, Harvard T.H. Chan School of Public Health, Boston, Massachusetts, United States of America\\

\smallskip

* Corresponding Author Email

\newpage

%%%%%%%%%%%%%%%%%%%%%%%%%%%%%%%%%%%%%%%%%%%%%%%%%%%%%%%%%%%%%%%%%%
%%%%%%%%%%%%%%%%%%%%%%%%%%%%%%%%%%%%%%%%%%%%%%%%%%%%%%%%%%%%%%%%%%
%%%%%%%%%%%%%%%%%%%%%%%%%%%%%%%%%%%%%%%%%%%%%%%%%%%%%%%%%%%%%%%%%%

\begin{appendix}

%%%%%%%%%%%%%%%%%%%%%%%%%%%%%%%%%%%%%%%%%%%%%%%%%%%%%%%%%%
%%%% Resetting figure and table count for the appendix %%%
%%%%%%%%%%%%%%%%%%%%%%%%%%%%%%%%%%%%%%%%%%%%%%%%%%%%%%%%%%
%\setcounter{figure}{0} \renewcommand{\thefigure}{A.\arabic{figure}}
%\setcounter{table}{0} \renewcommand{\thetable}{A.\arabic{table}}

\newpage

\tableofcontents

\newpage

%%%%%%%%%%%%%%%%%%%%%%%%%%%%%%%%%%%%%%%%%%%%%%%%%%%%%%%%%%%%%%%%%%
%%%%%%%%%%%%%%%%%%%%%%%%%%%%%%%%%%%%%%%%%%%%%%%%%%%%%%%%%%%%%%%%%%
%%%%%%%%%%%%%%%%%%%%%%%%%%%%%%%%%%%%%%%%%%%%%%%%%%%%%%%%%%%%%%%%%%

\section{Data}
\label{sec:org8802288}

% \subsection{Data}
% \label{sec:orga1d1a98}

%%%%%%%%%%%%%%%%%%%%%%%%%%%%%%%%%%%%%%%%%%%%%%%%%%%%%%%%%%%%%%%%%%
%%%%%%%%%%%%%%%%%%%%%%%%%%%%%%%%%%%%%%%%%%%%%%%%%%%%%%%%%%%%%%%%%%
%%%%%%%%%%%%%%%%%%%%%%%%%%%%%%%%%%%%%%%%%%%%%%%%%%%%%%%%%%%%%%%%%%

\subsection{Survey Data}
\label{sec:org09db5e8}

%%%%%%%%%%%%%%%%%%%%%%%%%%%%%%%%%%%%%%%%%%%%%%%%%%%%%%%%%%%%%%%%%%
%%%%%%%%%%%%%%%%%%%%%%%%%%%%%%%%%%%%%%%%%%%%%%%%%%%%%%%%%%%%%%%%%%

The study consisted of 33 sub-Saharan countries, which in approximate order from North to South and then West to East are: Senegal, Gambia, Guinea, Sierra Leone, Liberia, Mali, Burkina Faso, Côte d’Ivoire, Ghana, Togo, Benin, Niger, Nigeria, Cameroon, Chad, Ethiopia, Gabon, The Republic of the Congo (the Congo), The Demorcratic Republic of the Congo (DR Congo), Uganda, Kenya, Rwanda, Burundi, Tanzania, Angola, Zambia, Malawi, Mozambique, Zimbabwe, Namibia, Eswatini, Lesotho, South Africa". 
We excluded 4 countries: Equitorial Guinea, Guinea-Bissau, the Central African Republic and Botswana. These were excluded because they either had no surveys conducted, or, where surveys were available, there was insufficient information available on circumcision.  

We identified 109 nationally representative household surveys conducted the study region  between 2002 and 2019. 
These included several major survey series, namely the Demographic and Health Surveys (DHS), AIDS Indicator Surveys (AIS) Population-based HIV Impact Assessment (PHIA) surveys, Multiple Indicator Cluster Surveys (MICS), and, in the case of South Africa, Human Sciences Research Council (HSRC) surveys \textbf{[Citations]}. 
Data on self-reported circumcision status were available in 104 out of the total 109 surveys. 
\textbf{[Expand on reasons for exclusions]} [Link in your survey figure] Excluded surveys lacked information on age at circumcision, which made it impossible to make inferences on circumcision under our time-to-event modelling framework.  

\begin{figure}[H]
    \centering
    % \includegraphics[width=.9\linewidth]{plots/01_survey_table.pdf}
    \caption{Household surveys detailing circumcision patterns in SSA. The colour and size of points are determined by the provider and sample size of each respective survey. Triangular points have no information on circumcision type.}
    \label{fig:enter-label}
\end{figure}

Demographic variables such as age, residence and survey sampling weight were extracted from all surveys, alongside information related to circumcision: self-reported circumcision status, age at circumcision, Who performed the circumcision (whether the individual was circumcised by a health worker/professional, traditional practioner, other or unknown)  and Where did the circumcision take place (whether the individual was circumcised at a health facility, in the home of a health worker/professional, at home, at a ritual site, other or unknown). \textbf{[Write and link in survey table].}

% \begin{landscape}
{\linespread{1}
\small
\begin{longtable}[c]{ll ccc ccc ccc ccc}
      %%%%%%%%%%%%%%%%%%%%
      %%% Table header %%%
      %%%%%%%%%%%%%%%%%%%%
      % Upper layer of table header
      \hline
      % Lines inside the header
      \multicolumn{1}{l}{} & &  &  & \multicolumn{2}{c}{\bf Circumcision Type} \\ 
      \cmidrule(lr){5-7}
       & & {\bf Circumcision} & {\bf Age at} & {\bf Who performed} & {\bf Where was the}  \\
       & & {\bf Status} & {\bf Circumcision} & {\bf the circumcision?} & {\bf circumcision performed?} \\[5pt]
      % Lower lines
      \hline
      \vspace{-5pt}
      \endhead
      %%%%%%%%%%%%%%%%%%%%
      %%% Table footer %%%
      %%%%%%%%%%%%%%%%%%%%
      \\[-5pt]\hline
      \caption{Table caption}
      \endfoot
      %%%%%%%%%%%%%%%%%%%%%%
      %%% Table contents %%%
      %%%%%%%%%%%%%%%%%%%%%%
      \multicolumn{2}{l}{\textbf{Angola}} \\
      & 2015 (DHS) & \checkmark & \checkmark & \checkmark & \checkmark \\[5pt]
      \multicolumn{2}{l}{\textbf{Burundi}} \\
      & 2010 (DHS) & \checkmark & \checkmark & \checkmark & \checkmark \\ 
      & 2016 (DHS) & \checkmark & \checkmark & \checkmark & \checkmark \\[5pt]
      \multicolumn{2}{l}{\textbf{South Africa}} \\
      & 2002 (HSRC) &  \\
      & 2008 (HSRC) &  \\
      & 2012 (HSRC) &  \\
      & 2016 (DHS)  &  \\
      & 2017 (HSRC) &  
\end{longtable}
}
% \end{landscape}

The age of respondents was calculated from the century-month-code (CMC) of birth and interview dates. 
Survey respondents were located to their districts using masked cluster geocoordinates. 
Where these coordinates were unavailable (as in several MICS surveys) respondents were located to their level 1 administrative boundaries, most often corresponding to a province.  

Circumcisions were classed as a medical male circumcision (MMC) and traditional male circumcision (TMC) sing responses to both ‘Who performed the circumcision?’ and ‘Where did the circumcision take place?’ were classified \textbf{[Link to the table]}. 
Respondents that did not have any information related to the type of circumcision were classified as having an “unknown” male circumcision (MC).   

{\linespread{1}
\small
\begin{table}[htbp]
	\centering
	\caption{Criteria used to classify circumcisions from the survey data as MMC or TMC.}
	\label{tab::MCclassification}
	\begin{tabular}{c l | c c c}
	\multirow{2}{*}{} & & \multicolumn{3}{c}{\bf Who performed the circumcision?} \\
	                    &      & \begin{tabular}{c}\bf Healthcare\\[-5pt] \bf Professional\end{tabular} & \begin{tabular}{c}\bf Traditional \\[-5pt] \bf Practioner\end{tabular} & \begin{tabular}{c}\bf Missing\end{tabular} \\
	                          \hline
		% \multirow{3}{*}{\rotatebox{90}{Where performed?}}  & \begin{tabular}{l}Healthcare Facility\end{tabular} & MMC & MMC & MMC \\
		\multirow{3}{*}{\rotatebox{90}{\begin{tabular}{c}\bf Where was it\\[-5pt] \bf performed?\end{tabular}}}  & \begin{tabular}{l}\bf Healthcare\\[-5pt] \bf Facility \end{tabular} & MMC & MMC & MMC \\
		& \begin{tabular}{l}\bf Home, School \\[-5pt] \bf or Ritual Site \end{tabular} & MMC & TMC & TMC \\
		& \begin{tabular}{l} \bf Missing\end{tabular} & MMC & TMC & MC \\
	\end{tabular}	
\end{table}
}

In total, information was extracted from XXX, XXX respondents from all surveys.
\textbf{[Participation rates table]}

%%%%%%%%%%%%%%%%%%%%%%%%%%%%%%%%%%%%%%%%%%%%%%%%%%%%%%%%%%%%%%%%%%
%%%%%%%%%%%%%%%%%%%%%%%%%%%%%%%%%%%%%%%%%%%%%%%%%%%%%%%%%%%%%%%%%%

\subsection{Administrative boundaries}
\label{sec:org0995c95}

%%%%%%%%%%%%%%%%%%%%%%%%%%%%%%%%%%%%%%%%%%%%%%%%%%%%%%%%%%%%%%%%%%
%%%%%%%%%%%%%%%%%%%%%%%%%%%%%%%%%%%%%%%%%%%%%%%%%%%%%%%%%%%%%%%%%%

[Add in information about the boundaries] 

%%%%%%%%%%%%%%%%%%%%%%%%%%%%%%%%%%%%%%%%%%%%%%%%%%%%%%%%%%%%%%%%%%
%%%%%%%%%%%%%%%%%%%%%%%%%%%%%%%%%%%%%%%%%%%%%%%%%%%%%%%%%%%%%%%%%%

\subsection{Population}
\label{sec:org0d7d6e9}

%%%%%%%%%%%%%%%%%%%%%%%%%%%%%%%%%%%%%%%%%%%%%%%%%%%%%%%%%%%%%%%%%%
%%%%%%%%%%%%%%%%%%%%%%%%%%%%%%%%%%%%%%%%%%%%%%%%%%%%%%%%%%%%%%%%%%

Estimates for the male population size by district and five-year age group from 2002 through 2019 were sourced from XXX. 
Distributed (?) to single-year of age according to ??? 

%%%%%%%%%%%%%%%%%%%%%%%%%%%%%%%%%%%%%%%%%%%%%%%%%%%%%%%%%%%%%%%%%%
%%%%%%%%%%%%%%%%%%%%%%%%%%%%%%%%%%%%%%%%%%%%%%%%%%%%%%%%%%%%%%%%%%
%%%%%%%%%%%%%%%%%%%%%%%%%%%%%%%%%%%%%%%%%%%%%%%%%%%%%%%%%%%%%%%%%%

\section{Model}
\label{sec:org38d457a}

%%%%%%%%%%%%%%%%%%%%%%%%%%%%%%%%%%%%%%%%%%%%%%%%%%%%%%%%%%%%%%%%%%
%%%%%%%%%%%%%%%%%%%%%%%%%%%%%%%%%%%%%%%%%%%%%%%%%%%%%%%%%%%%%%%%%%
%%%%%%%%%%%%%%%%%%%%%%%%%%%%%%%%%%%%%%%%%%%%%%%%%%%%%%%%%%%%%%%%%%

{\color{red} \bf
To do:
\begin{itemize}
    \item Add in typeless model
    \item Aggregated areas
    \item Model calibration
    \item Prior choice
    \item Citations
\end{itemize}}

\noindent Following a similar approach to \cite{thomas2021multilevel}, we used a Bayesian hierarchical model with small area estimation methods to model probabilities of
circumcision. The Multi-level Model for Male Circumcision (threemc) uses survey data to estimates the probabilities and coverage of of circumcision were estimated using a competing risk discrete time-to-event model stratified by region, age, and time for two types of circumcision: (1) circumcisions that occurred in traditional male initiation ceremonies or for other religious or cultural reasons (TMC) and (2) circumcisions for non-traditional reasons and/or HIV prevention that take place in a clinical setting using medical methods (MMC). 

%%%%%%%%%%%%%%%%%%%%%%%%%%%%%%%%%%%%%%%%%%%%%%%%%%%%%%%%%%%%%%%%%%
%%%%%%%%%%%%%%%%%%%%%%%%%%%%%%%%%%%%%%%%%%%%%%%%%%%%%%%%%%%%%%%%%%

\subsection{Probabilities of circumcision (by type)}
\label{sec::tradcirc}

%%%%%%%%%%%%%%%%%%%%%%%%%%%%%%%%%%%%%%%%%%%%%%%%%%%%%%%%%%%%%%%%%%
%%%%%%%%%%%%%%%%%%%%%%%%%%%%%%%%%%%%%%%%%%%%%%%%%%%%%%%%%%%%%%%%%%

We used survey data to model the probabilities of recieving a TMC, MMC or MC of any time for individuals residing in region $i \in I = \{1, 2, \ldots, N_I\}$ at age $a \in A = \{0, 1, 2, \ldots, N_A\}$ and time $t \in T = \{1, 2, \ldots, N_T\}$. 

%%%%%%%%%%%%%%%%%%%%%%%%%%%%%%%%%%%%%%%%%%%%%%%%%%%%%%%%%%%%%%%%%%

\subsubsection*{Traditional circumcision}

%%%%%%%%%%%%%%%%%%%%%%%%%%%%%%%%%%%%%%%%%%%%%%%%%%%%%%%%%%%%%%%%%%

\noindent We define $\lambda^{\text{TMC}}_{iat}$ as the probability that an individual in region $i$ received TMC at age $a$ and time step $t$, given they were uncircumcised by age $a-1$ and time $t-1$, 
\begin{equation} 
		\lambda^{\text{TMC}}_{iat} = \mathbb{P}(\text{TMC} \; \text{in} \; (i,a,t) \; | \; \text{Uncircumcised in} \; (i,a-1, t-1)). 
	\label{eqn::TMC}
\end{equation}
A key assumption of \cite{thomas2021multilevel} model was that TMC was constant over time, due to traditional male initiation ceremonies amongst tribal, cultural and religious groups remained stable over time in South Africa but an assumption of a constant probability of TMC might not be feasible everywhere. For example, in some countries the level of TMC has changed through time due to demographic shifts and practices over time. There has also been a shift (or a replacement) of TMCs through MMCs, particularly in VMMC priority countries, to ensure HIV prevention efficacy and/or for safety considerations. There may also be cases where TMC remains stable and allowing a constant probability of TMC in the competing risks framework will cause a reduction in TMC coverage overall in places where VMMC uptake is high.

Therefore, we consider two models for $\lambda^{\text{TMC}}_{iat}$ in each country: (i) time invariant (i.e. constant over time) and (ii) time variant (i.e. non-constant over time). The {\it time-invariant} probability of TMC was modelled using a piece-wise logit-linear function, 
\begin{equation*} 
	\text{logit}(\lambda^{\text{TMC}}_{iat}) = \hat{\alpha} + \hat{\psi}_i + \hat{\phi}_a + \hat{\gamma}_{ia}
\end{equation*}
where $\hat{\alpha}$ is the intercept, $\hat{\psi_i}$ is a regional random effect, $\hat{\phi}_a$ is an age random effect, and $\hat{\gamma}_{ia}$ is an age-space interaction term to allow different age patterns of TMC across regions. The {\it time-variant} probability of TMC was also modelled using a piece-wise logit-linear function, 
\begin{equation*} 
	\text{logit}(\lambda^{\text{TMC}}_{iat}) = \hat{\alpha} + \hat{\psi}_i + \hat{\phi}_a + \hat{\theta}_t + \hat{\gamma}_{ia} + \hat{\delta}_{at}
\end{equation*}
where $\hat{\alpha}$ is the intercept, $\hat{\psi_i}$ is a regional random effect, $\hat{\phi}_a$ is an age random effect, $\theta_i$ is a temporal random effect,  $\hat{\gamma}_{ia}$ is an age-space interaction term to allow different age patterns of TMC  across regions and $\hat{\delta}_{at}$ is an age-time interaction term to allow for different age patterns of TMCs over time. We did not include a space-time interaction into due to computational reasons. 

The choice of whether a country used a time-invariant or time-variant probability of circumcision was decided through {\color{red}\bf[Need more information]}

%%%%%%%%%%%%%%%%%%%%%%%%%%%%%%%%%%%%%%%%%%%%%%%%%%%%%%%%%%%%%%%%%%

\subsubsection*{Medical circumcision}

%%%%%%%%%%%%%%%%%%%%%%%%%%%%%%%%%%%%%%%%%%%%%%%%%%%%%%%%%%%%%%%%%%

\noindent We defined $\lambda^{\text{MMC}}_{iat}$ as the probability an individual in region $i$ received MMC at age $a$ and time $t$, given they were uncircumcised by age $a-1$ and time $t-1$,  
\begin{equation}
		\lambda^{\text{MMC}}_{iat} = \mathbb{P}(\text{MMC} \; \text{in} \; (i,a,t) \; | \; \text{Uncircumcised in} \; (i,a-1, t-1)),
	\label{eqn::mmc}
\end{equation}

To ensure the overall probability of circumcision $\lambda^{\text{TMC}}_{iat} + \lambda^{\text{MMC}}_{iat} \leq 1$ for all $i$, $a$ and $t$ in the discrete-time framework, we assumed the probability of TMC occurred before MMC in each time step. Therefore, we practically model the probability an individual in region $i$ received MMC at age $a$ and time $t$, given they were uncircumcised by age $a-1$ and time $t-1$ and did not receive TMC at age $a$ and time $t$
\begin{equation*}
		\tilde{\lambda}^{\text{MMC}}_{iat} = \mathbb{P}(\text{MMC} \; \text{in} \; (i,a,t) \; | \; \text{Uncircumcised in} \; (i,a-1, t-1) \text{ and no TMC in} \; (i,a, t))
\end{equation*}
such that
\begin{equation*}
		\lambda^{\text{MMC}}_{iat} = \tilde{\lambda}^{\text{MMC}}_{iat}\cdot (1-\lambda^{\text{TMC}}_{iat}).
\end{equation*}

{\color{red}\bf[CITE]} included VMMC programmatic data in their model and programmes only provide MMCs for HIV prevention to those aged 10 and over, while infant and paediatric medical circumcision tends to occur through cultural or religious practices unrelated to scale-up of VMMC for HIV prevention. Therefore, to ensure any increase in MMC for those aged 10 and over did not affect the probability of MMC in those under 10, $\tilde{\lambda}^{\text{MMC}}_{iat}$ was separated into two processes: (i) paediatric circumcision (for those aged 0--9) and (ii) adolescent and adult circumcision (for those aged 10 and over). 

Here, we consider two models {\color{red}\bf[Need more information]} 

We modelled $\tilde{\lambda}^{\text{MMC}}_{iat}$ using a piece-wise logit-linear function:

\begin{equation*}
	\text{logit}(\tilde{\lambda}^{\text{MMC}}_{iat}) = \alpha + \psi_i + \phi_a + \theta_t + \gamma_{ia} + \delta_{at} + \zeta_{it} 
\end{equation*}
where $\alpha$ is the intercept, $\psi_i$ is a regional random effect, $\theta_i$ is a temporal random effect, $\phi_a$ is an age random effect, $\gamma_{ia}$ is an age-space interaction terms to allow for different patterns of MMCs over age between regions $\delta_{at}$ is an age-time interaction term to account for any changing age patterns in MMCs over time and $\zeta_{it}$ is an space-time interaction term to allow for different uptake of MMCs over time across regions. 


Therefore, we modelled $\tilde{\lambda}^{\text{MMC}}_{iat}$ using a piece-wise logit-linear function:
\begin{equation*}
	\text{logit}(\tilde{\lambda}^{\text{MMC}}_{iat}) =
	\begin{cases}
		\bar{\alpha} + \bar{\psi}_i + \bar{\phi}_a + \bar{\gamma}_{ia} & \text{for } 0 \leq a \leq 9\\
		\alpha + \psi_i + \phi_a + \theta_t + \gamma_{ia} + \delta_{at} + \zeta_{it} & \text{for } a \geq 10
	\end{cases} 
\end{equation*}
where $\bar{\alpha}$ and $\alpha$ are intercepts, $\bar{\psi}_i$ and $\psi_i$ are regional random effects, $\theta_i$ are temporal random effects, $\bar{\phi}_a$ and $\phi_a$ are age random effects, $\bar{\gamma}_{ia}$ and $\gamma_{ia}$ are age-space interaction terms to allow for different age patterns of MMCs between regions, for paediatric circumcision and adolescent and adult circumcision respectively. The $\delta_{at}$ is an age-time interaction term to allow for different age patterns of MMCs over time and $\zeta_{it}$ is an space-time interaction term to allow for different uptake of MMCs over time across regions, in adolescent and adult men. We assume that the probabilities of MMC among those aged 0--9 were constant over time, (i.e. $\lambda^{\text{MMC}}_{iat} \equiv \lambda^{\text{MMC}}_{ia}$ when $0\leq a \leq 9$), as paediatric medical circumcision practices are unrelated to VMMC for HIV prevention and are relatively stable. 

The choice of whether a country used a {\color{red}\bf[Need more information]} or {\color{red}\bf[Need more information]} probability of circumcision was decided through {\color{red}\bf[Need more information]}

%%%%%%%%%%%%%%%%%%%%%%%%%%%%%%%%%%%%%%%%%%%%%%%%%%%%%%%%%%%%%%%%%%

\subsubsection*{Male circumcision}

%%%%%%%%%%%%%%%%%%%%%%%%%%%%%%%%%%%%%%%%%%%%%%%%%%%%%%%%%%%%%%%%%%

\noindent We define $\lambda^{\text{MC}}_{iat}$ as the probability of circumcision of any type, at age $a$ and time $t$ given they were uncircumcised by age $a-1$ and time $t-1$ as
\begin{equation}
	\begin{split}
	\lambda^{\text{MC}}_{iat} &= \mathbb{P}(\text{MC} \; \text{in} \; (i,a,t) \; | \; \text{Uncircumcised in} \; (i,a-1,t-1)) \\ 
	              &= \lambda^{\text{TMC}}_{ia} + \lambda^{\text{MMC}}_{iat}. 
	\end{split}
	\label{eqn::tot}
\end{equation}

%%%%%%%%%%%%%%%%%%%%%%%%%%%%%%%%%%%%%%%%%%%%%%%%%%%%%%%%%%%%%%%%%%
%%%%%%%%%%%%%%%%%%%%%%%%%%%%%%%%%%%%%%%%%%%%%%%%%%%%%%%%%%%%%%%%%%

\subsection{Circumcision coverage}

%%%%%%%%%%%%%%%%%%%%%%%%%%%%%%%%%%%%%%%%%%%%%%%%%%%%%%%%%%%%%%%%%%
%%%%%%%%%%%%%%%%%%%%%%%%%%%%%%%%%%%%%%%%%%%%%%%%%%%%%%%%%%%%%%%%%%

The probability $S_{iat}$ of remaining uncircumcised up to age $a$ and time $t$,  typically referred to as the survivor function, was expressed as
\begin{equation}
	\begin{split}
		S_{iat} &= \mathbb{P}(\text{Uncircumcised in} \; (i,a-1,t-1)) \\
				&= \prod_{(0,(t-a))}^{(a-1,t-1)}\lambda^{\text{UC}}_{iat} \\
				&= \lambda^{\text{UC}}_{i,0,t-a}\cdot \lambda^{\text{UC}}_{i,1,t-a+1}\cdot\ldots \cdot\lambda^{\text{UC}}_{i,a-1,t-1}
	\end{split}
	\label{eqn::survfunc}
\end{equation}
where $\lambda^{\text{UC}}_{iat}$ is the probability an individual in region $i$ remained uncircumcised at age $a$ and time $t$, given they were uncircumcised by age $a-1$ and time $t-1$, was
\begin{equation*}
	\begin{split}
		\lambda^{\text{UC}}_{iat} &= \mathbb{P}(\text{Uncircumcised in}  \; (i,a,t) \; | \; \text{Uncircumcised in} \; (i,a-1, t-1)),\\
		&= 1 - \lambda^{\text{MC}}_{iat} 
	\end{split}
	\label{eqn::uncirc}
\end{equation*}

The cumulative incidence function (CIF) defines the marginal probability (or coverage/proportion) of individuals who received a TMC, a MMC or a MC of any type by age $a$ and time $t$, accounting for the competing risk of other circumcision types. This was calculated as the sum of incidence of circumcision by type $k$ at each age up to $a$:
\begin{equation*}
	\begin{split}
		\text{CIF}_{iat}^k &= \mathbb{P}(k \; \text{by} \; (i,a,t)) \\
		&= \sum_{(0,(t-a))}^{(a,t)} I^k_{iat}
	\end{split}
	\label{eqn::cuminc}
\end{equation*}
where $I^k_{iat}$ is the incidence of circumcision of type $k\in \{\textrm{TMC}, \textrm{MMC}, \textrm{MC}\}$, in region $i$, at age $a$ and time $t$, 
\begin{equation*}
	\begin{split}
		I_{iat}^k &= \mathbb{P}(k \; \text{in} \; (i,a,t)) \\
		       &= \mathbb{P}(k \; \text{in} \; (i,a,t) | \; \text{Uncircumcised in} \; (i,a-1,t-1))\times \\
		       & \;\;\;\;\;\;\;\;\;\;\;\;\;\;\;\;\;\mathbb{P}(\text{Uncircumcised in} \; (i,a-1,t-1)) \\
	    	   &= \lambda_{iat}^k \cdot S_{iat} 
	\end{split}
	\label{eqn::inc}
\end{equation*}

%%%%%%%%%%%%%%%%%%%%%%%%%%%%%%%%%%%%%%%%%%%%%%%%%%%%%%%%%%%%%%%%%%
%%%%%%%%%%%%%%%%%%%%%%%%%%%%%%%%%%%%%%%%%%%%%%%%%%%%%%%%%%%%%%%%%%

\subsection{Model priors}

%%%%%%%%%%%%%%%%%%%%%%%%%%%%%%%%%%%%%%%%%%%%%%%%%%%%%%%%%%%%%%%%%%
%%%%%%%%%%%%%%%%%%%%%%%%%%%%%%%%%%%%%%%%%%%%%%%%%%%%%%%%%%%%%%%%%%

In space, we assigned the random effects $\hat{\psi}_i$, $\bar{\psi}_i$, $\psi_i$, $\hat{\gamma}_{ia}$, $\gamma_{ia}$ and $\zeta_{it}$, intrinsic conditional autoregressive (ICAR) priors in space \cite{besag1995conditional}. An ICAR model allows us to encodes spatial dependence between neighbouring regions which allows information on the probability and coverage of circumcision in areas where data is sparse or non-existent to be borrowed. For a generic parameter $\beta_i$, an ICAR model assumes that the mean of $\beta_i$ is a weighted average its neighbours, 
\begin{equation*}
	\beta_i \;|\; \beta_{j}, j \sim i, \tau_{\beta} \sim \text{N}\left(\frac{1}{n_i} \sum_{j \sim i} \beta_j, \frac{1}{n_i\tau_{\beta}} \right) \;\;\; i = 1, 2, 3,\ldots, N_I
\end{equation*}
Here, $j \sim i$ refers to the neighbours of region $i$, $n_i$ is the number of neighbours and $\tau_{\beta}$ is the marginal precision \cite{rue2005gaussian}. 

In age, we modelled the random effects ($\hat{\phi}_a$, $\bar{\phi}_a$, $\phi_a$, $\hat{\gamma}_{ia}$, $\bar{\gamma}_{ia}$, $\gamma_{ia}$ and $\delta_{at}$) using penalised B-splines. For a generic parameter $\beta_a$, penalised B-splines assumes that $\beta_a$ is a sum of basis functions, 
\begin{align*} 
	\beta_a = \sum_{j = 1}^{J} w_{j}b_{aj}
\end{align*} 
where $b_{aj}$ are B-spline basis functions with knots placed every five years (at 0, 5, 10 etc.), $J$ is the number of splines used and $\omega_{j}$ are the spline weight parameters to be estimated. The weights $w_{j}$ are penalised using an AR1 process 
\begin{align*} 
  w_{j} \; | \; w_{j-1}, \rho_{w}, \tau_{w} &\sim N(\rho_{w} \cdot w_{j-1}, \;\; \tau^{-1}_{w}) \;\;\;  j = 2, 3,\ldots, J \\
  w_{1} \; | \; \rho_{w}, \tau_{w} &\sim N(0, \tau^{-1}_{w}) 
\end{align*}
where $|\rho_{w}| < 1$ is an autocorrelation parameters controlling the correlation of the effect of the current age on the previous age and $\tau_{w}$ is the precision.

In time, we consider three possible prior distirbution choices: an autoregressive process of order 1 (AR1), random walk of order 1 (RW1) and random walk of order 2 (RW2). {\color{red} \bf [Need more information]}
% \noindent In time, we assigned the random effects $\theta_t$, $\delta_{at}$ and $\zeta_{it}$ an autoregressive process of order 1 (AR1) priors,
\begin{align*} 
  \beta_t \; | \; \beta_{t-1},  \rho_{\beta}, \tau_{\beta} &\sim
  \begin{cases}
  	N(0, \tau^{-1}_{\phi}) & t = 1\\
  	N(\rho_{\beta} \cdot \beta_{t-1}, \;\; \tau^{-1}_{\beta}) \;\;\; & t = 2, 3,\ldots, T
  \end{cases}
\end{align*}

\begin{align*} 
  \beta_t \; | \; \beta_{t-1}, \rho_{\beta}, \tau_{\beta} &\sim
  \begin{cases}
  	N(0, \tau^{-1}_{\phi}) & t = 1\\
  	N(\beta_{t-1}, \;\; \tau^{-1}_{\beta}) \;\;\; & t = 2, 3,\ldots, T
  \end{cases}
\end{align*}

\begin{align*} 
  \beta_t \; | \; \beta_{t-1}, \beta_{t-2}, \rho_{\beta}, \tau_{\beta} &\sim
  \begin{cases}
  	N(0, \tau^{-1}_{\phi}) & t = 1, 2\\
  	N(2\beta_{t-1} - \beta_{t-2}, \;\; \tau^{-1}_{\beta}) \;\;\; & t = 3, 4,\ldots, T
  \end{cases}
\end{align*}

% where $|\rho_{\phi}| < 1$ is an autocorrelation parameters controlling the correlation of the effect of the current age on the previous age and $\tau_{\beta}$ is the precision. 

Moreover, age-space, age-time, and space-time interaction terms ($\hat{\gamma}_{ia}$, $\bar{\gamma}_{ia}$, $\gamma_{ia}$, $\delta_{at}$ and $\zeta_{it}$) are modelled using a Type IV interactions as defined by Knorr-Held and Leonard \cite{knorr2000bayesian},

We assigned Gaussian priors to each of the intercepts, $\alpha, \;\hat{\alpha}, \;\bar{\alpha} \sim N(0, 5^2)$. Exponential priors were placed on standard deviations, $\sigma_{(\cdot)} = \sqrt{\tau^{-1}_{(\cdot)}} \sim \text{Exp}(1)$. Gaussian priors were set for all correlation parameters on the logit scale, such that 
\begin{align*} 
  \hat{\rho}_{(\cdot)} = \frac{2}{1 + \exp(-\rho_{(\cdot)})} - 1 \sim N(3, 1^2)
\end{align*}
corresponding to 95\% CI prior weight that the autocorrelation parameters are between 0.48 and 0.99 on the real scale.

%%%%%%%%%%%%%%%%%%%%%%%%%%%%%%%%%%%%%%%%%%%%%%%%%%%%%%%%%%%%%%%%%%
%%%%%%%%%%%%%%%%%%%%%%%%%%%%%%%%%%%%%%%%%%%%%%%%%%%%%%%%%%%%%%%%%%

\subsection{Likelihood}

%%%%%%%%%%%%%%%%%%%%%%%%%%%%%%%%%%%%%%%%%%%%%%%%%%%%%%%%%%%%%%%%%%
%%%%%%%%%%%%%%%%%%%%%%%%%%%%%%%%%%%%%%%%%%%%%%%%%%%%%%%%%%%%%%%%%%

Survey data consists of individual observations of self-reported age at circumcision, date of birth, and circumcision type (medical or traditional) from male survey respondents. Circumcision status observations were right censored if the respondent reported not being circumcised at the time of the survey or left censored if the individual reported being circumcised at the time of survey but did not report their age at circumcision.\\

\noindent Each survey $s \in S = \{1, 2, \ldots, N_S\}$ consists of sampled individuals $j \in J_s = \{0, 1, 2, \ldots, N_{J_s}\}$ residing in region $i \in I = \{1, 2, \ldots, N_I\}$. Surveyed individuals were followed from their year of birth until their year of circumcision or their year of censoring. For circumcised individuals, the year of circumcision was calculated as the year of birth plus the age at circumcision. It was assumed that no circumcisions occurred after age 59, so for uncircumcised individuals, censoring occurred either in the survey year or when they were 59. Individuals who self-reported they were circumcised but have a missing age at circumcision were included in the analysis through left censoring. \\

\noindent The likelihood function for each individual is this defined by the following groups:
\begin{itemize}
	\item[(1)] \textbf{TMC observed (TMC)}: Individuals in region $i$ reporting an TMC at age $a$ and time $t$. The likelihood of this outcome was the probability an individual in region $i$ reported a TMC at age $a$ and time $t$, $I^{\text{TMC}}_{iat} = \lambda^{\text{TMC}}_{iat}\cdot S_{iat}$. 
	\item[(2)] \textbf{MMC observed (MMC)}: Individuals in region $i$ reporting an MMC at age $a$ at time $t$. The likelihood of this outcome was the incidence of MMC in region $i$, at age $a$ and time $t$, $I^{\text{MMC}}_{iat} = \lambda^{\text{MMC}}_{iat}\cdot S_{iat}$. 
	\item[(3)] \textbf{MC observed (MC)}: Individuals in region $i$ reporting they were circumcised but had a missing or unknown type at age $a$ and time $t$. The likelihood of this outcome was the incidence of MC in region $i$, at age $a$ and time $t$, $I^{\text{MC}}_{iat} = \lambda^{\text{MC}}_{iat}\cdot S_{iat}$. 
	\item[(4)] \textbf{Right censored (RC)}: Individuals in region $i$ who remained uncircumcised by age $a$ at time $t$. The likelihood of this outcome was the probability of remaining uncircumcised in region $i$, at age $a$ and time $t$,$S_{iat}$
	\item[(5)] \textbf{TMC left censored (TMCLC)}: Individuals in region $i$ reporting a TMC at an unknown age and time between birth at time $t-a$ and age $a$ at time $t$. The likelihood of this outcome was the probability of receiving a TMC in region $i$ before age $a$ at time $t$, $\text{CIF}^{\text{TMC}}_{iat}$
	\item[(6)] \textbf{MMC left censored (MMCLC)}: Individuals in region $i$ reporting a MMC at an unknown age and time between birth at time $t-a$ and age $a$ at time $t$. The likelihood of this outcome was the probability of receiving a MMC in region $i$ before age $a$ at time $t$, $\text{CIF}^{\text{MMC}}_{iat}$ 
	\item[(7)] \textbf{MC left censored (MCLC)}: Individuals in region $i$ reporting a circumcision, of unknown type at an unknown age and time, between birth at time $t-a$ and age $a$ at time $t$. The likelihood of this outcome was the probability of being circumcised in region $i$ before age $a$ at time $t$, $(1-S_{iat}) = \text{CIF}^{\text{MMC}}_{iat}$ 
\end{itemize}

\noindent Taken together, the partial likelihood for the discrete time-to-event model may be expressed as 
\begin{eqnarray*}
	L(\boldsymbol{\Theta}) &= \displaystyle\prod_{(i,a,t)} \prod_{s=1}^{N_s} \prod_{j=1}^{J_s}  \underbrace{\left(\lambda^{\text{TMC}}_{iat}\cdot S_{iat}\right)^{\mathbbm{1}^{\text{TMC}}_{sjiat}}}_{\text{(1)}} \cdot  \underbrace{ \left(\lambda^{\text{MMC}}_{iat} \cdot S_{iat}\right)^{\mathbbm{1}^\text{MMC}_{sjiat}}}_{\text{(2)}} \cdot\underbrace{\left(\lambda^{\text{MC}}_{iat} \cdot S_{iat}\right)^{\mathbbm{1}^\text{MC}_{sjiat}}}_{\text{(3)}} \cdot \\
 & \hspace{1cm}  \underbrace{\left(S_{iat}\right)^{\mathbbm{1}^{RC}_{sjiat}}}_{\text{(4)}} \cdot \underbrace{\left(\text{CIF}^{\text{TMC}}_{iat}\right)^{\mathbbm{1}^{\text{TMCLC}}_{sjiat}}}_{\text{(5)}} \cdot \underbrace{\left(\text{CIF}^{\text{MMC}}_{iat}\right)^{\mathbbm{1}^{\text{MMCLC}}_{sjiat}}}_{\text{(6)}} \cdot \underbrace{\left(\text{CIF}^{\text{MC}}_{iat}\right)^{\mathbbm{1}^{\text{MCLC}}_{sjiat}}}_{\text{(7)}} \\
	   &= \displaystyle\prod_{(i,a,t)} 
    \left(\lambda^{\text{TMC}}_{iat}\cdot S_{iat}\right)^{\text{N}^{\text{TMC}}_{iat}} \cdot  
    \left(\lambda^{\text{MMC}}_{iat} \cdot S_{iat}\right)^{\text{N}^\text{MMC}_{iat}} \cdot 
    \left(\lambda^{\text{MC}}_{iat} \cdot S_{iat}\right)^{\text{N}^\text{MC}_{iat}}  \cdot \\ & \hspace{1cm}  \left(S_{iat}\right)^{\text{N}^{RC}_{iat}} \cdot \left(\text{CIF}^{\text{TMC}}_{iat}\right)^{\text{N}^{\text{TMCLC}}_{iat}} \cdot \left(\text{CIF}^{\text{MMC}}_{iat}\right)^{\text{N}^{\text{MMCLC}}_{iat}}\cdot 
    \left(\text{CIF}^{\text{MC}}_{iat}\right)^{\text{N}^{\text{MCLC}}_{iat}}
\end{eqnarray*}
where $\mathbbm{1}^{l}_{sjiat}$ is an indicator variable indicating whether individual $j$ in survey $s$ and region $i$ reported outcome $l$ at age $a$ and time $t$ and $\text{N}^{l}_{sjiat}$ denotes the total number of individuals reporting outcome (l) in region $i$ at age $a$ and time $t$.
\begin{equation*}
	\text{N}^{l}_{iat} = \sum_{s=1}^{N_s}\sum_{j=1}^{J_s} \mathbbm{1}^k_{sjiat}
\end{equation*}
where $l$ is TMC, MMC, MC, RC, MMCLC, TMCLC or MCLC which correspond to the outcomes (1)-(7) above.\\

\noindent Survey data are collected through a complex two-stage cluster sampling design with unequal sampling probabilities. Performing model inference using a standard likelihood may lead to biased estimates so to account for these survey designs, the probabilities of circumcision were estimated using a weighted pseudo-likelihood in which we replaced the observed counts $\text{N}^l_{iat}$ with weighted counts $\tilde{\text{N}}^l_{iat}$, calculated using survey weights. Each individual $j$ in survey $s$ residing in region $i$ had sampling weight $\omega_{sji}$ which were normalised using the Kish effective sample size, 
\begin{equation*}
	\tilde\omega_{sji} = \frac{\omega_{sji}}{\bar{\omega}_{si}}\cdot \frac{M_s}{M^{\text{eff}}_s} 
\end{equation*}
Here, $\bar{\omega}_{si}$ is the arithmetic mean of all survey weights from the individuals sampled in survey $s$ and region $i$ and $M_s$ is the total sample size in survey $s$ and $M^{\text{eff}}_s$ is the Kish effective sample size which accounts for heterogeneity in sampling weights and is calculated using 
\begin{equation*}
	M^{\text{eff}}_s = \frac{(\sum_j \omega_{sji})^2}{\sum_j \omega_{sji}^{2}}.
\end{equation*}
Using the normalised sampling weights, $\tilde\omega_{sji}$, we calculate the weighted counts, $\tilde{\text{N}}^l_{iat}$, for each region, age and time stratum using 
\begin{equation*}
	\tilde{\text{N}}^{l}_{iat} = \sum_{s=1}^{N_S}\sum_{j=1}^{N_{J_s}}\tilde\omega_{sji} \mathbbm{1}^{\text{l}}_{sjiat}.
\end{equation*}
where $l$ is TMC, MMC, MC, RC, MMCLC, TMCLC or MCLC which correspond to the outcomes (1)-(7) above. 

%%%%%%%%%%%%%%%%%%%%%%%%%%%%%%%%%%%%%%%%%%%%%%%%%%%%%%%%%%%%%%%%%%
%%%%%%%%%%%%%%%%%%%%%%%%%%%%%%%%%%%%%%%%%%%%%%%%%%%%%%%%%%%%%%%%%%

\subsection{Model checking}

%%%%%%%%%%%%%%%%%%%%%%%%%%%%%%%%%%%%%%%%%%%%%%%%%%%%%%%%%%%%%%%%%%
%%%%%%%%%%%%%%%%%%%%%%%%%%%%%%%%%%%%%%%%%%%%%%%%%%%%%%%%%%%%%%%%%%



%%%%%%%%%%%%%%%%%%%%%%%%%%%%%%%%%%%%%%%%%%%%%%%%%%%%%%%%%%%%%%%%%%
%%%%%%%%%%%%%%%%%%%%%%%%%%%%%%%%%%%%%%%%%%%%%%%%%%%%%%%%%%%%%%%%%%

\subsection{Inference}

%%%%%%%%%%%%%%%%%%%%%%%%%%%%%%%%%%%%%%%%%%%%%%%%%%%%%%%%%%%%%%%%%%
%%%%%%%%%%%%%%%%%%%%%%%%%%%%%%%%%%%%%%%%%%%%%%%%%%%%%%%%%%%%%%%%%%








% \emph{Put a general discussion element for the model giving an overview}

% "threemc"("Multi-level model for male circumcision") is a Bayesian, spatio-temporal, competing-risks, time-to-event model (reference). 
% We have extended this model from its initial application in South Africa to 33 countries within the SSA region. 
% The model produces estimates of circumcision rates, incidence and coverage (i.e. cumulative indidence), with associated uncertainty bounds, stratified by type, year, age and location.
% Circumcision rates are projected after their most recent household survey assuming continuation of estimated age-specific rates, with probabilistic uncertainty. 
% Estimates for single ages were binned into 5-year age groups from 0-4 to 54-59, and other age groups of interest, such as the VMMC target age group of 10-29 year-olds.  

% \emph{Traditional Circumcision}

% A key assumption of the original threemc model was that TMC was constant over time, due to the perceived intransigence in TMC practices and traditional male initiation ceremonies amongst tribal, cultural and religious groups which have developed over time. 
% MMCs amongst paediatric males, defined as those under 10, were also assumed to be constant over time, as VMMC programmes, the main force behind the adoption of MMC in much of SSA, do not targets males below 10 (reference). 

% [Technical details] 

% Explain the TMC with time and TMC with no time.  

% \emph{Medical Circumcision}

% Explain paed cut off and no paed cut off 

% \subsection{Model Specification}
% \label{sec:org40f2f33}

% In our choice of model specification, we were interested in two main assumptions/features of the
% model:
% \begin{itemize}
% \item How we should treat TMC, in terms of whether to continue to assume a constant rate of TMC over time, or to reject that assumption,
% \item How to model paediatric MMC, which should be minimal in at least the VMMC target countries.
% \end{itemize}

% If possible, we hoped to use the same model for every country, or failing that, come up with a
% satisfactory choice of model specification which would make sense both qualitatively and
% quantitatively. 

% Qualitatively, we have made some presumptions about certain countries and their circumcision patterns.
% In non-VMMC SSA countries, concentrated in Western and Central Africa (WCA), TMC has historically made up the bulk of MCs.
% Therefore, most MMCs in non-VMMC countries are likely to have superseded TMCs performed as part of traditional male initiation ceremonies. This suggests that MMCs in these countries are likely to be on paediatric individuals in traditional settings, so the assumption of constant and
% negligible paedaitric MMC could be a poor one. 
% Because any increases in MMC will come at the expense of TMC in our "competing-risks" model, it is also plausible that the assumption that TMC rates in these countries have been relatively constant may be unrealistic.
% It is therefore likely that the inclusion of a time effect for TMC and not partitioning MMC into adult and time-invariant paediatric rates will be a more realistic reflection of circumcision
% patterns in non-VMMC countries.  

% Conversely, in VMMC priority countries changes in circumcision patterns have largely been driven by the intervention of VMMC programme implementation. 
% As such, it is more realistic to assume that paediatric MMCs are minimal, in line with UNAIDS VMMC policy.
% It is more difficult to say whether the rate of TMC will be constant over time in non-VMMC countries.
% Historical TMC patterns in these countries, differ significantly, even subnationally.
% It may therefore be more realistic to also allow TMC to vary over time in VMMC countries.
% In countries where the rate of TMC is stable over time, the model will capture this behaviour, while in countries where TMC varies over time, the inclusion of a time effect for TMC will provide the model with the additional required flexibility to identify this trend.
% We would also prefer to not have to treat every VMMC priority country separately with regards to their model specification, so including a time effect for TMC in the models for these countries seems like a logical choice. 

% We have also performed a quantitative analysis of the different model specifications available
% to us. A more detailed treatment of this can be found in section x of the appendix. 

% For each country, we fit a model for each possible model specification for threemc, that is:
% \begin{itemize}
% \item for each choice of temporal prior, namely AR1, RW1 and RW2,
% \item a choice of whether or not to include a paediatric age cutoff for MMC of ten, and
% \item a choice of whether to include a temporal effect for TMC.
% \end{itemize}
% Thus totalling 12 possible models for each country.
% These models were fit to all of the survey data available, rather than a subset of the data.
% This was because we were most interested here in seeing how the inclusion of a peadiatric age cutoff and/or a temporal TMC effect would effect the fit of the model to the data available, rather than in the relative short-term forecasting ability of the different specifications.
% As the choice of temporal prior was more important when forecasting, we were not especially interested in how each of these performed here, but regardless we fit for each temporal prior, for completeness. 

% The survey weighted empirical survey circumcision coverage was compared to a sample of 1000 draws from the posterior predictions of circumcision coverage for each country.
% Both were aggregated to area level 1 and to five year age groups (from 15-19 through 55-59), to avoid having too many zeros in the survey estimates.

% Comparisons were made of mean predictions, using expected log-posterior density (ELPD) and continuous ranked probability scores (CRPS), as well as error statistics such as the mean error (ME), root mean error (RME) and root mean error squared (RMSE).
% Also evaluated was the the "calibration" of our model with regards to it's posterior predictive uncertainty.
% This involved comparing survey estimates of circumcision coverage with the 50\%, 80\% and 95\% credible intervals (CIs) of our posterior predictive distribution. A "good" calibration was regarded as one in which roughly 50\% of survey observations fell within the 50\% CI range, 85\% within the 85\% CI range, and 95\% within the 95\% range.
% Seeing as we were comparing within-sample mean estimates, we were particularly interested in the RMSE of our predictions compared to the survey estimates. 

% Just as we distinguished between VMMC and non-VMMC countries when considering the qualitative merits of each model specification, so too did a pattern emerge here when quantitively comparing the model fits for both sets of SSA countries. 

% For the non-VMMC countries there was a drastic difference in mean predictive accuracy between the different model specifications.
% The best model specification was the model which included a temporal TMC effect, but not a paediatric age cutoff for MMC.
% This specification performed the best for x / y countries, averaging a RMSE of x, CRPS of x and ELPD of x, in comparison to the next best specification, ?, which averaged an RMSE of x, CRPS of x and ELPD of x. 
% It appeared that the inclusion of a temporal TMC effect was very influential in improving the fit of the model for non-VMMC countries, with specifications including this term averaging a RMSE of x versus x for those which did not include this term.
% Models including an MMC paediatric age cutoff of 10 consistently performed worse than other models, averaging an RMSE of x versus x. 
% These findings were in line with our previous intuitions on MC patterns in non-VMMC, mainly WCA countries, where TMC was historically high, and increases in MMC are likely within the previously TMC population, and hence MMCs were likely performed on younger individuals than in VMMC countries. 

% In contrast, for the VMMC priority countries, the choice of whether to include an MMC paedaitric age cutoff had little effect on model fit.
% This may be because under fifteens were not surveyed, and so there were no survey estimates for paediatric populations to compare to our model predictions.
% However, the inclusion of this MMC paediatic age cutoff did have a negative effect to the fit for the models to adults for non-VMMC countries, so the fact that it's inclusion here did not hurt model fit suggests it was a more accurate representation of MMC patterns in VMMC priority countries.
% As we were aware of VMMC programme policy in not currently circumcising under fifteens, and since it did not seem to negatively effect model fit to adults, we decided to include a paediatric age cutoff of 10 in the model for VMMC coutries. 

% Models which including a temporal TMC effect were marginally better, with an average RMSE of x versus x for all other models.
% In particular, Kenya, which had high historical TMC in most regions and was the most similar of the VMMC countries to WCA countries in terms of it's circumcision patterns, saw a marked improvement when this effect was included, with an average RMSE of x versus x. For completeness, we decided to include a temporal effect for TMC in the model specification for VMMC priority countries.

% It was therefore ultimately decided to include a temporal effect for TMC in the model for both VMMC and non-VMMC countries, while only including a pedaitric age cutoff for VMMC countries.
% A table with the full set of fit statistics for each model specification for each country can be found in section x of the appendix.

% \subsection{Model Calibration and Choice of Temporal Prior}
% \label{sec:orgbbc65cb}

% For some VMMC priority countries, we did not have access to more recent survey data. 
% One particular country where this is the case is Tanzania, whose most recent survey was the 2016 PHIA survey.
% In these circumstances, there may have been a significant increase in MMC coverage due to VMMC programme implementation which was not captured within our survey period.
% VMMC programme data was an available source of more recent circumcision data.
% The DMPPT2 model explicitly used this data to estimate MMC. 
% The results of DMPPT2, as well as those for countries with more recent surveys who have experienced significant MMC scaleups, suggest that VMMC may have scaled up at a rate not anticipated by threemc where only these older surveys are available.
% This was consistent with out-of-sample (OOS) exploration of model fits to countries like Zimbabwe, where removing access to the most recent (2018 DHS) survey similarly underestimates VMMC scale up.
% Hence, it was felt that threemc likely underestimates uncertainty with regards to predicting circumcision coverage for progressively later years from our last available surveys, particularly in the case of VMMC priority countries, which have seen rapid increases in historically low circumcision.
% A more dramatic "fanning" out of our prediction interval as we forecast further from the last available survey data was therefore deemed desirable, consistent with having greater uncertainty in our future forecasts. 

% The two main drivers of uncertainty over time in threemc were:
% \begin{itemize}
% \item The variance hyperparameters relating to time, including the variance hyperparameters for space-time and age-time interactions, and
% \item The choice of temporal prior, for which threemc supports the use of an AR1, RW1 or RW2 prior.
% \end{itemize}
% The "unpooled" optimised time-related variance hyperparameters for the model fit for each respective country varied significantly, but in general certain patterns and values for these hyperparameters could be associated with a having larger bounds for successive prediction years.

% Due to computational constraints, we could not model the entire SSA region together as one singular area hierarchy, which, through partial pooling and the neighbourhood correlation structure inherent in the model, would allow the model to borrow information from countries with a large amount of available data to inform predictions in countries with older and/or fewer surveys.
% One alternative to using a partially-pooled model was to use the uncertainty estimates which produce the best predictions for countries with more recent data to inform our uncertainty estimates in countries with less recent survey data available.
% To quantitatively explore this hypothesis, we performed an OOS evaluation of the model fit to each country, removing their most recent survey data (or, in the case where there were two surveys in subsequent years, the two most recent surveys) and comparing posterior predictions to the survey-estimated circumcision coverage, as we did with our analysis of different model specifications.
% A grid search over sensible variance hyperparameter values from the "naive" threemc fits for each country and temporal prior was employed, to determine the optimal values and temporal prior choice. 
% For the AR 1 model, the effect of different time correlation parameters on our uncertainty bounds was determined to be minimal, and in the interests of parsimony, these parameters were ignored in our calibration efforts with this model.

% TODO: Finish this write up!
% TODO: Add results! Also add ternary plot 



%%%%%%%%%%%%%%%%%%%%%%%%%%%%%%%%%%%%%%%%%%%%%%%%%%%%%%%%%%%%%%%%%%
%%%%%%%%%%%%%%%%%%%%%%%%%%%%%%%%%%%%%%%%%%%%%%%%%%%%%%%%%%%%%%%%%%
%%%%%%%%%%%%%%%%%%%%%%%%%%%%%%%%%%%%%%%%%%%%%%%%%%%%%%%%%%%%%%%%%%

\end{appendix}

%%%%%%%%%%%%%%%%%%%%%%%%%%%%%%%%%%%%%%%%%%%%%%%%%%%%%%%%%%%%%%%%%%
%%%%%%%%%%%%%%%%%%%%%%%%%%%%%%%%%%%%%%%%%%%%%%%%%%%%%%%%%%%%%%%%%%
%%%%%%%%%%%%%%%%%%%%%%%%%%%%%%%%%%%%%%%%%%%%%%%%%%%%%%%%%%%%%%%%%%

\end{document}

%%%%%%%%%%%%%%%%%%%%%%%%%%%%%%%%%%%%%%%%%%%%%%%%%%%%%%%%%%%%%%%%%%
%%%%%%%%%%%%%%%%%%%%%%%%%%%%%%%%%%%%%%%%%%%%%%%%%%%%%%%%%%%%%%%%%%
%%%%%%%%%%%%%%%%%%%%%%%%%%%%%%%%%%%%%%%%%%%%%%%%%%%%%%%%%%%%%%%%%%
